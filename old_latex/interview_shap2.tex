\subsection{Interview version SHAP2}
\subsubsection*{Introduction}
In this interview, we want to understand which explanations are the best (which is: most understandable and most correct). We will ask you to create some explanations yourself and give you explanations from the computer that you need to evaluate. This information will eventually be used to learn a computer to evaluate its own explanations.

You may have heard about artificial intelligence (AI) or machine learning on the news or from other media outlets. In short, machine learning (which is a big part of artificial intelligence) is a way for computers to learn to make predictions based on data. They make these predictions by finding patterns in the data. For example, imagine we want to make the computer predict whether a piece of fruit is an apple or a banana, based on data about the colour and shape of the fruit. The computer can learn that bananas are yellow and elongated, and apples can have multiple colours and are round.

This is still quite simple and easy to understand, but in practice, machine learning can be used for very complex predictions. To understand why a machine learning model makes the predictions it does, we can let the computer explain its predictions. These explanations are important when problems are not completely formalized (it is not precisely clear what the machine learning model has to do) and when the consequences of a wrong decision are critical (for example in health care).

In this interview, we focus on natural language processing: a type of machine learning where the data is text. More specifically, for this interview, we "taught" a computer to predict whether a Twitter post is about floods (inundation) or not. The computer explains its prediction by giving us the words on which it based its prediction. During this interview, imagine you are part of a disaster relief team. Imagine there has just been a flood in Queensland (a state in Australia) and your task is to get information about the flood from social media. You run the computer program that classifies Twitter text as relevant to the flood or not. The computer gave you predictions of the relevance of sentences and explanations of those predictions. In this interview, we are going to test the quality of the explanations. This may seem irrelevant, but understandable and accurate explanations are important. For example, when the model explains itself well, we can have more trust in it.

\subsubsection*{Informed consent}
Depending on the participant's preference, the interview will be recorded for more accurate information retrieval. These recordings may be heard by the interviewer and their supervisors. The recordings will be deleted within eight months after the interview. Participants may be quoted directly. Personal information that will be documented contains experience with machine learning and experience with floods and flood-related text.

Participants are allowed to skip any question and can decide to stop their participation at any time, for any reason.

If there are any unclarities, always feel free to ask for clarification.

The thesis can be downloaded from https://www.cs.ru.nl/bachelors-theses/ once it has been completed.

\subsubsection*{Personal details}
\begin{itemize}
    \item What is your current experience with machine learning?
    \item What is your current experience with floods and flood-related text?
\end{itemize}

\subsubsection*{Questions}
\begin{enumerate}
    \item We just gave you a description of what an explanation would look like in our application. When would an explanation as described in the introduction be a good explanation? Or phrased in a different way: what are the characteristics of a good explanation, according to your intuition?

    \item To see how humans would make such an explanation, we are putting you in the computer's place. We will you a sentence and your task is to predict whether this sentence is about floods or not and explain your prediction. Keep in mind that for some of the sentences, the prediction is obvious.
    First I will give an example. Take the sentence "Queensland counts flood cost as New South Wales braces for river peaks." I would say it is about floods, because of the words "flood" and "river".
    \begin{itemize}
        \item "Road, garden and driveway officially flooded"
        \item "Queensland does not deserve this flood trauma \& nightmare again. I hate it for them."
        \item "Insurance Talk: Queensland flood claims exceed \$44.8 million, likely to hit \$52.1 million soon"
        \item "Harsh country delivers harsh lessons: Fire, flood and other natural disasters are part of life in Australia"
    \end{itemize}

    \item Here will follow a few sets of explanations generated by a computer. Your task is to select the best explanation from each set. Of course, you are able to give comments on all explanations (for example if you like or dislike all of them). \begin{enumerate}
        \item Set 1: explanations that fit all principles except one: \begin{enumerate}
            \item Fitting all. \begin{itemize}
                \item Sentence to be explained: "Daring rescue of teenager from flood waters: Australia reels from surging floods";
                \item Prediction: relevant;
                \item Explanation: "rescue", "floods".
            \end{itemize}
            \item No fidelity. \begin{itemize}
                \item Sentence to be explained: "New South Wales braces for river peaks as Queensland counts flood cost";
                \item Prediction: relevant;
                \item Explanation: "braces", "cost".
            \end{itemize}
            \item No compactness. \begin{itemize}
                \item Sentence to be explained: "Baby stuffed in bag, hoisted from flood - Australia's powerful storms led to an amazing rescue of two women and a baby";
                \item Prediction: relevant;
                \item Explanation: "hoisted", "flood", "powerful storms", "amazing rescue", "baby", "stuffed", "bag".
            \end{itemize}
            \item No distinctiveness. \begin{itemize}
                \item Sentence to be explained: "Gladstone flood victims returning home: Floodwaters are receding at Gladstone, in central Queensland";
                \item Prediction: relevant;
                \item Explanation: "flood".
            \end{itemize}
            \item No contrast. \begin{itemize}
                \item Sentence to be explained: "If all you people are thirsty come drink some of the flood waters in Queensland";
                \item Prediction: relevant;
                \item Explanation: "drink", "waters".
            \end{itemize}
        \end{enumerate}
        
        \item Set 2: explanations that fit no principles except one: \begin{enumerate}
            \item Fitting none. \begin{itemize}
                \item Sentence to be explained: "Daring rescue of teenager from flood waters: Australia reels from surging floods";
                \item Prediction: relevant;
                \item Explanation: "Daring", "teenager", "reels", "surging".
            \end{itemize}
            \item Fidelity. \begin{itemize}
                \item Sentence to be explained: "New South Wales braces for river peaks as Queensland counts flood cost";
                \item Prediction: relevant;
                \item Explanation: "river", "peaks", "flood", "cost".
            \end{itemize}
            \item Compactness. \begin{itemize}
                \item Sentence to be explained: "Baby stuffed in bag, hoisted from flood - Australia's powerful storms led to an amazing rescue of two women and a baby";
                \item Prediction: relevant;
                \item Explanation: "amazing"
            \end{itemize}
            \item Distinctiveness. \begin{itemize}
                \item Sentence to be explained: "Gladstone flood victims returning home: Floodwaters are receding at Gladstone, in central Queensland";
                \item Prediction: relevant;
                \item Explanation: "Gladstone", "returning", "receding".
            \end{itemize}
            \item Contrast. \begin{itemize}
                \item Sentence to be explained: "If all you people are thirsty come drink some of the flood waters in Queensland";
                \item Prediction: relevant;
                \item Explanation: "people", "thirsty", "flood", "waters".
            \end{itemize}
        \end{enumerate}
        
        \item Set 3: explanations that fit some principles: \begin{enumerate}
            \item Fidelity, contrast, distinctiveness. \begin{itemize}
                \item Sentence to be explained: "Gladstone flood victims returning home: Floodwaters are receding at Gladstone, in central Queensland";
                \item Prediction: relevant;
                \item Explanation: "flood", "victims", "returning", "floodwaters", "receding".
            \end{itemize}
            \item Distinctiveness, compactness. \begin{itemize}
                \item Sentence to be explained: "Australia seeks Army's help to tackle flood crisis, thousands evacuate";
                \item Prediction: relevant;
                \item Explanation: "army", "thousands".
            \end{itemize}
        \end{enumerate}
    \end{enumerate}

    \item When you understand an explanation, you are more likely to remember it. We will show you an explanation. Your task is to remember the explanation while you are doing another task. \begin{itemize}
        \item Sentence to be explained: "Not long ago we were driving around Queensland marvelling at roadside poles showing high-water levels in the previous flood. Mind-boggling."
        \item Prediction: relevant;
        \item Explanation: "high", "water", "levels", "flood"
    \end{itemize}
    Now double the number 2 until the sum of the digits is 14 or larger.
    Do you remember the explanation (first with, then without the sentence?
    \todo{Write down which principles these explanations fit}
    
    \item And another explanation that you need to remember during another task. \begin{itemize}
        \item Sentence to be explained: "Queensland floods: 3 dead, thousands isolated: Three people are now dead as Queensland’s flood crisis escalates."
        \item Prediction: relevant;
        \item Explanation: "floods", "dead", "crisis".
    \end{itemize}
    
    \item From your experience during this interview, can you give your considerations in: \begin{enumerate}
        \item Would you say it is important for an explanation to be close to reality? [fidelity]
        \item Would you say explanations need to be complete, meaning they need to explain many cases? [completeness]
        \item Would you say short explanations are generally better than long ones? [compactness]
        \item Would you say that, when explaining specific events, unique characteristics of that event should be used? [distinctiveness]
        \item Would you say that explanations of a prediction should mention what would have to change to have a different prediction? [contrast]
    \end{enumerate}
\end{enumerate}